\documentclass[11pt,a4paper]{article}

\usepackage[pdftex]{color,graphicx}
\usepackage[utf8]{inputenc}
\usepackage{siunitx}
\usepackage{geometry}
\newgeometry{margin=2.0cm}
\usepackage{hyperref}

\hypersetup{
    colorlinks=true,
    linkcolor=blue,
    urlcolor=blue
}

\usepackage[backend=biber,style=authoryear,sorting=nyt,dashed=false]{biblatex}
\renewcommand*{\nameyeardelim}{\addcomma\space}
\addbibresource{references/references.bib} % note the .bib is required

\newcommand*\mean[1]{\overline{#1}}
\newcommand\todo[1]{\textbf{TODO: #1}}

\title{SUPPLEMENTARY: Effects of Vertical Shear on Cloud Field Variability and Organization }
\author{Mark Muetzelfeldt}
\date{June 2017}

\begin{document}

\maketitle

\section{UM}

Running a vanilla copy of the UM, vn10.7 (which includes the idealized code, plus the Smagorinsky bugfix). I think the correct paper to cite for this is: \cite{walters2017met}. This paper describes GA6.0/6.1. See this table: \href{https://code.metoffice.gov.uk/trac/GA/wiki/GAJobs#UMjobsforGA6.0GL6.0GC2.0}{GA6 table, suites}. N.B. the most recent version it mentions is vn10.3, the only GA versions that mention vn10.7 are GA7.0. There is no equiv. paper for GA7.0 yet, the only ref I can find is in a paper by this paper, which mentions that a paper by Walters et al. is in prep: \href{https://doi.org/10.1051/0004-6361/201630020}{Exploring the climate of Proxima B with the Met Office Unified Model}.

\section{Suite}

Initially, the main suite used was \href{https://code.metoffice.gov.uk/trac/roses-u/browser/a/l/0/0/0/trunk}{u-al000}. Revisions 42978:42985 were used for the bulk of the modelling, between 9/6/2017 - 11/6/2017.

For the actual analysis, the main suite used was \href{https://code.metoffice.gov.uk/trac/roses-u/browser/a/n/3/8/8/trunk}{u-an388}. Revision 43539 was used for the bulk of the modelling, between 17/6/2017 - 18/6/2017.

\section{Omnium and analysis}

Analysis was done using \href{https://github.com/markmuetz/omnium}{omnium} and \href{https://github.com/markmuetz/scaffold_analysis}{scaffold\_analysis}. 

\section{Energy loss}

\begin{itemize}
    \item Cooling rate: $C(z)$; [\SI{}{K.s^{-1}}]
    \item Energy loss (rate): $E_{loss} = \int_{0}^{TOA} \rho c_p C(z) dz$; [\SI{}{W.m^{-2}}]
    \item Hydrostatic equation: $\frac{dp}{dz} = - \rho g$
    \item Therefore:
    \item Energy loss (rate): $E_{loss} = - \frac{c_p}{g} \int_{p_{TOA}}^{p_0} C(z) dz$; [\SI{}{W.m^{-2}}]
\end{itemize}

\section{Extra figs}

\begin{figure}[htp!]
    \centering
    \includegraphics[width=400px]{figs/{atmos.profile_plot.dz_profile}.png}
    \caption{dz with height}
    \label{fig:dz_profile}
\end{figure}

\begin{figure}[htp!]
    \centering
    \includegraphics[width=400px]{figs/{atmos.mass_flux_plot.z0_both}.png}
    \caption{mf hist z0}
    \label{fig:mf_hist_z0}
\end{figure}

\begin{figure}[htp!]
    \centering
    \includegraphics[width=400px]{figs/{atmos.mass_flux_plot.z1_both}.png}
    \caption{mf hist z1}
    \label{fig:mf_hist_z1}
\end{figure}

\begin{figure}[htp!]
    \centering
    \includegraphics[width=400px]{figs/{atmos.mass_flux_plot.z2_both}.png}
    \caption{mf hist z2}
    \label{fig:mf_hist_z2}
\end{figure}

\subsection{Momentum transports}

I have dropped this from the main paper. The reason? I am having difficulty interpreting it and it doesn't add to much to the specifics of organization or variability. It does show I think why the wind doesn't relax back to the reference profile at \SI{12}{km}, but I might just explain this and leave this plot here.
\begin{itemize}
    \item The momentum flux can be calculated using $\rho \mean{u' v'} = \rho \frac{u_{inc}}{\Delta t} \Delta z$.
    \item \todo{keep?} Figure \ref{fig:momf_profile}: Momentum flux profiles for different experiments (c.f. \cite{RE2001}, Fig. 7).
\end{itemize}

\begin{figure}[h]
    \centering
    \includegraphics{figs/{atmos.profile_plot.momentum_flux_profile}.png}
    \caption{}
    \label{fig:momf_profile}
\end{figure}


\section{Reference locations}

\begin{enumerate}
    \item \cite{wing2017convective}: Section 3.2, p14.
\end{enumerate}


\printbibliography[title={References}]


\end{document}