\documentclass[11pt,a4paper]{article}

\usepackage[pdftex]{color,graphicx}
\usepackage[utf8]{inputenc}
\usepackage{siunitx}
\usepackage{geometry}
\usepackage{float}
\newgeometry{margin=2.5cm}

\usepackage[backend=biber,style=authoryear,sorting=nyt,dashed=false]{biblatex}
\renewcommand*{\nameyeardelim}{\addcomma\space}
\addbibresource{references/references.bib} % note the .bib is required

\newcommand*\mean[1]{\overline{#1}}
\newcommand\todo[1]{\textbf{TODO: #1}}

\title{Effects of Vertical Shear on Cloud Field Organization and Variability}
\author{Mark Muetzelfeldt, Robert Plant and  Peter Clark}
\date{June 2017}

% General notes:
% Be quantitative
% Do not make woolly comparisons
% Think about (sampling) errors 
% Be clear about speculation
% If speculating, ask yourself why you haven't made a better case for something
% Think about what evidence you need to support a conclusion - e.g. difference between distn integral vs 99th percentile


% Glossary:
% CRM: Cloud Resolving Model
% LCL: Lifting Condensation Level
% PDF: Probability Density Function
% OCF: Optimized Conservative Filter
% QE: Quasi-Equilibrium
% RCE: Radiative-Convective Equilibrium
% SISL: Semi-Implicit, Semi-Lagrangian
% UM: Unified Model

% Thoughts:
% Change relationship to distribution?
% Change experiment to case?

% TODOs:

% DONE:
% replaced 1.5, 2, 2.5 km with actual heights: 1657, 2062, 2518 m.

% Misspellings:
% mass-flux
% parameterization
% parameterisation
% parametrisation

% Go to phrases (AKA tics):
% in particular
% slightly

\begin{document}

\maketitle

\begin{center}
\section*{Abstract}
\end{center}
\todo{write abstract}


\section{Introduction}

%\begin{itemize}
%    \item Justification for using RCE - links to QE.
%    \item Motivation for how this work fits into larger picture of stochastic parametrization. Particularly the variability analysis.
%    \item Mention: organization and momentum flux.
%\end{itemize}
Most convection parametrization schemes in use are deterministic - for a given grid-column state they will produce a single specific profile of convective heating, moistening and perhaps momentum transport. This can be justified through the Quasi-Equilibrium (QE) assumption of \cite{arakawa1974interaction}, in which it is assumed that the convection is in local equilibrium with the grid-scale forcing because each grid-column contains many convective plumes. As model resolution increases, this assumption will break down, as each grid-column contains fewer and fewer convective plumes. One way of modelling the parametrized convection in these circumstances is by using a statistical physics stochastic parametrization scheme \parencite{berner2017stochastic} - the magnitude of the subgrid convection is now determined by sampling from a Probability Density Function (PDF) instead of being uniquely determined by the grid-scale state.

% \todo{mention sampling error}
% How to include sampling error? 
Statistical stochastic parametrization schemes require a characterization of the subgrid variability of the convection. This can be done by using Cloud Resolving Models (CRMs) to verify theoretical relationships of the statistical properties of the convection. In this study, we use a CRM to look at how the organization of deep convection affects the variability of the convection. In particular, we look at how squall line like features generated by deep vertical shear in the troposphere modify the distribution of convective mass flux.

%\todo{theoretical relationships: mean mass flux per cloud p(m), total mass flux: p(M), CC2006i (6) and (14). I'd like to introduce these here, along with explanation that these are what I would like to test, with the tentative hypothesis that org will affect the 2nd by increasing the variability at a given scale. What should I say about the 1st? On looking at the data, I think that increased lifetime is leading to larger overall clouds that persist longer. This appears as a decrease in lower mf clouds, and increase in larger, still described by an exponential decrease. Should this go after next para? }

Convective mass flux plays an important role in many convection parametrization schemes. The magnitude of the mass flux is typically determined through the use of a closure based on the grid-column state of the model. A popular choice of closure is the Convective Available Potential Energy (CAPE) closure assumption, whereby a percentage of the CAPE is assumed to be consumed by the convection over a given convective time scale. 
This is typically achieved by the use of a cloud model, which determines the profile of heating and moistening of the convection. The mass flux is then chosen so that the heating and moistening predicted by the cloud model consume the required amount of CAPE in a given timestep. This could be done by, for example, iterating from a first estimate of the mass flux to a value which removes the required amount of CAPE.
% a la Kain-Fritsch.
%Orig: This can be used to set the mass flux by iterating towards the required value necessary to remove the desired amount of CAPE. Through the use of a cloud model, the magnitude of the mass flux then determines the profile of heating and moistening of the atmosphere that will happen because of the unresolved convection.
% BP's comment: suggests that schemes to closure & then vertical profile. Some do, but for most it is calculated the other way round!

The CRM simulations carried out in this study are run to statistical equilibrium, in the sense that the model state taken over a large enough domain and time period is statistically indistinguishable from another such period. Due to the domain size of the model being large compared with the typical area of a convective cloud, the model as a whole is representative of the QE assumption as it contains many convective clouds. However, at smaller spatial or temporal scales, there will be significant departures from this mean state. It is precisely these fluctuations, and their relative frequencies, that we wish to characterize.

A theoretical relationship for the PDF of mass flux per cloud, $m$, was derived in \cite{CC2006I}. Under assumptions of weakly interacting clouds in statistical equilibrium, they showed that the PDF of $m$ decreased exponentially with increasing $m$:
\begin{equation}
    p(m) = \frac{1}{\langle m \rangle} \exp \left(\frac{-m}{\langle m \rangle} \right),
    \label{eqn:pdf_mass_flux_per_cloud}
\end{equation}
where angle brackets denote an ensemble average. They again use the weakly interacting clouds assumption to say that the number of clouds in a given region will be an independent variable, and can therefore be represented by a Poisson distribution. This, along with Equation \ref{eqn:pdf_mass_flux_per_cloud}, leads to a PDF for the total convective mass flux in a given region, $M$:
\begin{equation}
    p(M) = \sqrt{\frac{\langle N \rangle}{\langle m \rangle}} \exp \left( -\langle N \rangle M^{-\frac{1}{2}} \right) \exp \left( \frac{-M}{\langle m \rangle} \right)  I_1\left(2 \sqrt{\frac{\langle N \rangle}{\langle m \rangle} M}\right).
    \label{eqn:pdf_mass_flux}
\end{equation}
Here, $\langle N \rangle$ denotes the ensemble average number of clouds in a given region. $I_1(x)$ is the modified Bessel function of order 1.

When these theoretical relationships are compared to distributions taken from CRM simulations, they are found to hold up well \parencite{CC2006II}. It is however necessary to correct for the finite size of clouds when comparing Equation \ref{eqn:pdf_mass_flux} to numerically modelled distributions. They also apply deep shear across the domain, and find that the the shear ``has only a small effect on the mass flux statistics'', even though the shear causes organization of the cloud field. 
% \todo{tenses} OK? - shifted all to present tense
It is this finding that we investigate in more detail, through the use of systematically varying shear profiles and performing a comparison between the mass flux statistics for the different profiles. 
%We look for the transition from unorganized to organized behaviour and perform a comparison between the mass flux statistics for the different shear profiles. 
Specifically, we test how the mass flux per cloud is affected by increasing the cloud field organization by increasing the shear. Also, we look at how the statistics of the distribution of $M$ change as the organization increases.

Understanding how the convective mass flux fluctuates around a mean value can be used to build a statistical physics stochastic parametrization scheme. For deep convection, this idea was used by \cite{PC2008}, and for shallow convection this has been done by \cite{sakradzija2016stochastic}. The basic ideas for a stochastic deep convection scheme were outlined in \cite{PC2008}, and are reproduced here:

\begin{enumerate}
    \item Average the atmospheric state over a region large enough to contain many clouds (i.e. over a region for which the QE assumption is valid);
    \item Compute the equilibrium statistics of the full convective ensemble;
    \item Sample randomly from the equilibrium distribution to get the convective mass flux for each grid-column;
    \item Use these properties to compute convective tendencies, as in a standard convection\\parametrization scheme.
\end{enumerate}

It is our goal that the work in this study will lead to a better understanding of \textit{how} the equilibrium distribution needed for steps 2 and 3 varies with the vertical shear in the domain. This could lead to modifications of a stochastic convection parametrization scheme so that it can represent some of the effects of vertical shear. In particular, it should provide a route to including some of the effects of spatial organization into the scheme. 

\subsection{Organization in CRMs}
CRMs can exhibit different modes of organization. For example, when run with interactive radiation, CRMs can display convective self-aggregation \parencite{wing2017convective}. 
This is where convection develops in a large subregion of the domain, often related to the domain size of the model, whilst outside of this region compensating subsidence occurs. Many mechanisms for this have been found, and may be different from model to model. 
However, longwave radiation feedback has been found to be essential for its formation \parencite{wing2017convective}. As we wish to isolate the organization stimulated by shear, we use a uniform prescribed cooling profile over our domain, thus removing longwave radiation feedbacks and ensuring that convective self-aggregation will not occur.

%\todo{Vertical shear and its role in org. Difference between LLS/Deep Shear, evidence for both. Set out the fact I will be using deep. Vertical shear as a means of mom transport. Inclusion of this in p13n.}

Vertical wind shear has long been known to play a vital role in forming squall lines \parencite{TMM1982, RKW1988}. In this study, we use a deep shear profile, associated with observed organization in the Marshall Islands experiment \parencite{yanai1973determination}, as a means of stimulating organization in our experiments. This profile is varied systematically, in a manner similar to a previous study by \cite{RE2001}. In their case however, when examining deep shear, they chose to look at changes in the direction of shear, whereas we are concerned only with varying the strength of unidirectional shear.

% I can't think of how to bring this up without seeming to shoe-horn it in. Some of the ideas have been addressed in the above para.
%\todo{Stoch. p13n/QES: talk about why QES/RCE is a good way of running an RCE expt for use in a stoch. p13n. How you can use the results, the inverse model etc.}

%The theoretical relationships allow us the characterize how mass flux will vary. Mass flux is a key parameter in convection parametrization schemes, as it value is used to determine the behaviour of a cloud model. This will typically lead to a moistening of the upper troposphere through detrainment from the cloud, as well as a heating of the troposphere, primarily through the compensating subsidence. The magnitude of the mass flux is determined through the use of a closure based on the grid-column state of the model. A popular choice of closure is the Convective Available Potential Energy (CAPE) closure assumption, whereby a percentage of the CAPE is assumed to [be consumed] by the convection in a given timestep. This can be used to set the mass flux by iterating towards the required value or by theoretical arguments. The magnitude of the mass flux then determines how much heating and moistening of the atmosphere will happen because of the unresolved convection, through the use of a cloud model.

\newpage
\section{Model and experimental design}

\subsection{Model configuration}

%\begin{itemize}
%    \item Using an idealized version of the Met Office Unified Model (version 10.7).
%    \item Use resolutions of \SI{1}{km}.
%    \item Running over a domain of 256$\times$256 \SI{}{km^2}, for a duration of \SI{20}{days}. This should be large enough to cover the spatial extent of the squall lines, and long enough to both reach equilibrium and provide a long enough time to average over many squall lines' lifetimes.
%    \item Forced with prescribed cooling of \SI{2}{K.day^{-1}} - typical of radiative cooling rates over the tropics.
%    \item Using a 3D Smagorinsky Turbulence scheme, no convection scheme. Includes the OCF MC scheme.
%    \item Applying a relaxation back to a neutral profile above \SI{12.5}{km} (or \SI{15}{km}?).
%    \item Set out experimental design: choice of shear profiles and explanation of why these produce organization, linking back to previous work, e.g. \cite{RKW1988}.
%\end{itemize}

% Model, resolution, domain.
The model used is the UK Met Office Unified Model (UM), version 10.7 \parencite{walters2017met}, run in its idealized configuration. It is a fully compressible, non-hydrostatic model, and its dynamics are modelled using a Semi-Implicit, Semi-Lagrangian (SISL) advection scheme. It is run using Cartesian coordinates, over a bi-periodic domain of 256$\times$256 \SI{}{km^2}, using a horizontal resolution of \SI{2}{km} and a \SI{30}{s} timestep. It uses a variable vertical resolution taken from the Large Eddy Model \parencite{petch2001sensitivity}, starting at \SI{63.22}{m} at the surface, increasing to \SI{250}{m} by a height of \SI{2.5}{km}. The resolution increases again above \SI{15}{km}, increasing to \SI{1}{km} by \SI{25}{km} where it remains up to the top of the model at \SI{40}{km}. The change in vertical resolution with height below \SI{25}{km} can be seen in Figure \ref{fig:input_profiles} (leftmost figure).
The experiments are run for \SI{40}{days}, enough time for the simulations to reach equilibrium (see Figure \ref{fig:energy_fluxes}) and for sufficient statistics about the model state to be gathered. 

No convection parametrization scheme is used, instead the convection is explicitly modelled by the dynamics and the subgrid parametrization schemes of the model that are in use. The model uses a 3D Smagorinsky subgrid turbulence mixing scheme \parencite{smagorinsky1964some}, with no blending between this and the boundary layer scheme. It includes prognostic ice and graupel in its microphysics scheme \parencite{wilson1999microphysically}. The large-scale cloud is parametrized using the \cite{smith1990scheme} scheme. To obtain realistic vertical profiles in the boundary layer, the scheme described in \cite{lock2000new} is used. In a change to the normal configuration, the stability function from the Large Eddy Model \parencite{petch2001sensitivity} is used.
% todo: how to justify this choice of stability function? Best I could say ATM is "to match previous studies done using the LEM". Which is a bit lame.
The surface is treated as ocean, with a fixed temperature of \SI{300}{K} and a friction roughness length of \SI{0.0002}{m}. The surface heat and moisture fluxes are modelled using bulk aerodynamic formulae, using the same roughness length of \SI{0.0002}{m}.

All simulations are run with a Coriolis parameter, $f$, equal to 0. They are run without geostrophic forcing.

\subsection{Initial and cooling profiles}

Each simulation is started with an initial moisture and potential temperature profile that was designed to be a reference profile for the Large Eddy Model \parencite{petch2001sensitivity}, with a uniform $u$ wind of \SI{5}{m.s^{-1}}. These profiles are not the equilibrium profiles for our experiments, and so we allow the simulations to spin-up until they have reached their equilibrium states. They are allowed to spin-up for \SI{20}{days} while the surface fluxes come into balance with the energy and precipitation losses from the model (see Figure \ref{fig:energy_fluxes}). Output from the model is only analysed after the initial spin-up period.

A prescribed cooling rate of \SI{2}{K.day^{-1}} is applied from the surface to \SI{200}{hPa}, decreasing linearly with pressure to \SI{0}{K.day^{-1}} by \SI{100}{hPa} (see Figure \ref{fig:input_profiles}, middle). This cooling rate is comparable to typical radiative cooling rates over the tropics. Larger cooling rates representing large-scale convergence and forced ascent, ranging from \SI{4}{K.day^{-1}} to \SI{16}{K.day^{-1}}, were also tested. However, when the model was forced with these rates, it developed a large-scale, domain wide oscillation, thought to be a convectively coupled gravity wave with wavelength equal to the largest scale available to the disturbance - the length of the domain. These oscillations are a form of organization, and so affect the fluctuations that we are trying to measure. They are also an artefact of the modelling process due to the bi-periodic nature of the boundary conditions, and so are little more than a curiosity and an annoyance. We therefore decided to run with smaller cooling rates that did not stimulate this mode of organization.

\todo{Paragraph on stratospheric GW damping and theta relaxation.}

\subsection{Shear profiles and experiments}

The shear profile that we choose to impose was originally derived from the Marshall Islands experiment \parencite{yanai1973determination}. It has subsequently been used to stimulate mesoscale organization in the form of linear features resembling tropical squall lines, in both 2D \parencite{tompkins2000impact} and 3D \parencite{grabowski1996long, CC2006II} experiments. It is applied by relaxing back to a mean wind across the domain, by applying a single increment to all grid-cells at a given height level in the domain. In the east-west direction, the increment at each height are given by: $u_{inc} = \frac{U_{ref} - \langle u \rangle}{\tau} \Delta t$, where $U_{ref}$ is our reference profile, $\langle u \rangle$ is the domain mean $u$ wind. The relaxation time, $\tau = $ \SI{21600}{s}, and the model timestep, $\Delta t = $ \SI{30}{s}. 

%\todo{S0-S4 - linear scalings of the basic profile.}
The way that we vary the shear profile is to take linear scalings of the original profile. We do this as a pragmatic means for altering the degree of organization in the experiments. The scalings are labelled according to the strength of their shear profile - in S0 the relaxation is back to zero mean wind across the domain, in S4 the relaxation is back to the the same profile as was used in \cite{CC2006II, tompkins2000impact}. This allows us to begin to sample the parameter space of shear profiles, in a way that includes both unorganized (S0), and organized (S4) behaviour, with the degree of organization in the other experiments to be determined by this study.
% BP: the same as observed by Yanai et al? todo: MM: don't have an answer yet.

The reference profiles, along with the mean wind from the model over the final day of the simulation for each of the experiments, are shown in Figure \ref{fig:input_profiles}.

\begin{figure}[h]
    \centering
    \includegraphics[width=400px]{figs/{atmos.profile_plot.input_profiles}.png}
    \caption{The left figure shows the vertical resolution, $\Delta z$, as a function of height. The centre figure shows the prescribed cooling profile for S0. Note the cooling profile is specified in terms of pressure, not height, so the profiles from  different experiments are not all identical when plotted against height. They are very similar however. The right figure shows the reference $u$ wind profile, as well as the mean $u$ wind over the full domain averaged over the final day of each simulation, S0-S4. }
    \label{fig:input_profiles}
\end{figure}

\subsection{Moisture and energy balance}

The SISL advection scheme allows for a large timestep without becoming unstable. Such schemes are not conservative; there is no guarantee that there will be the same total quantity of a tracer from one timestep to the next. In preliminary tests, we found that the non-conservation of moisture had a large impact on the total water content in the model, which leads to precipitation rates being three times higher than what would be expected through comparison with evaporation rates. To ameliorate this, we used the Optimized Conservative Filter (OCF) moisture conservation scheme \parencite{zerroukat2015monotonic}, developed previously for this purpose. This scheme is similar to the Priestley scheme \parencite{priestley1993quasi}, in that it compares the total water content at two consecutive timesteps and ensures conservation by adding or removing water vapour as necessary from grid-cells. It does this in a way that is consistent with the uncertainty caused by using interpolation in the SISL scheme, using linear and higher order interpolation as bounds within which it can modify the water vapour in a grid-cell, while staying as close as possible to the higher order solution. With the OCF scheme in use, moisture conservation is guaranteed and once the model has reached equilibrium, precipitation is close to balancing evaporation as it should (see Figure \ref{fig:energy_fluxes}). When averaged over the final \SI{20}{days} of the experiment, the difference between the two is around 0.5\%. For S0, precipitation = \SI{151.0}{W.m^{-2}}, latent heat flux = \SI{150.4}{W.m^{-2}} when precipitation is expressed as its heating rate through latent heat release. This implies that the model must be losing a small amount of moisture at this time (not tested).
% todo: estimate the CWV loss, even better derive from model.

\begin{figure}[H]
    \centering
    \includegraphics[width=400px]{figs/{atmos.surf_flux_plot.energy_fluxes}.png}
    \caption{Latent heat flux (solid), sensible heat flux (dash-dot) and heating due to precipitation (dashed) over the run length of the model for experiments S0 and S4. Other experiments lie between these two. Precipitation data are smoothed using a one day averaging window. There is a large initial transient during the model spin-up. Results and statistics are only gathered after day 20 (vertical black dashed line), when the model is judged to have reached a statistical equilibrium. }
    \label{fig:energy_fluxes}
\end{figure}

In a similar manner to moisture, the SISL scheme can also lead to a non-conservation of energy. No suitable scheme for energy conservation is available in the UM, and so the effects of this energy non-conservation must be characterized and taken into account. This manifests itself as an imbalance between the prescribed cooling energy loss, and the energy input into the model through the surface latent and sensible heat fluxes. Essentially, energy is being supplied by the dynamics of the model, which means that the surface fluxes are less than what would be expected to balance the prescribed cooling. The energy loss from the prescribed cooling can be calculated by integrating the cooling over the pressure levels over which it is applied. This yields an energy loss rate of \SI{204.8}{W.m^{-2}} for S0, and \SI{206.1}{W.m^{-2}} for S4, with the difference being mainly due to a difference in the surface pressure. The combined sensible and latent heat fluxes are \SI{167.5}{W.m^{-2}} for S0, and \SI{119.7}{W.m^{-2}} for S4 (See Figure \ref{fig:energy_fluxes}). This means that the energy imbalance rate is \SI{37.3}{W.m^{-2}} for S0 and \SI{86.4}{W.m^{-2}} for S4.

For the purposes of this study, in which we wish to study the dynamical interactions between clouds, the energy imbalance is not an issue. This is supported by the fact that the imbalance mainly affects the stratosphere, i.e. the heating effects are far from the area of interest in this study (not shown). However, if a detailed energy budget were the focus of the investigation, this would become of prime importance and it would be necessary to either account for the imbalance rigorously, or to apply a similar scheme for energy conservation as the OCF scheme described above. 

\subsection{Cloud diagnosis}
\label{subsec:cloud_diag}

%\todo{description of how clouds defined for analysis, plus some mention of sensitivity to thresholds.}
We require a diagnosis of where the clouds are in 3D space. Choices for how to diagnose clouds includes using thresholds of cloud liquid water \parencite{CC2006II}, or updraught velocity \parencite{CC2006II, zipser1980cumulonimbus}. We use a thresholding technique based on both cloud liquid water and updraught velocity at the same time. The reason for this is that we would like to measure only updraughts for the purposes of mass flux calculations, hence the updraught threshold. We would also like to exclude vertical velocity fluctuations caused by gravity wave propagation, for which we use the cloud liquid water threshold. In line with the previous studies, we use values of $w =$ \SI{1}{m.s^{-1}} as our updraught threshold, and cloud liquid water = \SI{5e-3}{g.kg^{-1}}. Sensitivity to these thresholds was considered: we varied each threshold by 10\% lower and higher. The results were found to be qualitatively consistent with the analysis carried out below, although the precise numbers for each analysis of course varied slightly (not shown).

%\todo{how clouds are counted}
To work out how many clouds there are in the domain, it is necessary to use the information provided by the thresholding and find the contiguous areas where the thresholds are exceeded. This is done at each height level separately, i.e. there is no information about how diagnosed clouds at one height level are related to clouds on the adjacent levels. Grid-cells with diagnosed cloud are considered contiguous, and therefore part of the same cloud, if either their edge or their corner is in contact with another grid-cell with diagnosed cloud. This allows for diagnosis of how many clouds there are at a given level, and from this the mass flux per cloud can be calculated.
%\begin{itemize}
%    \item Profiles are applied by relaxing back to a mean wind across the domain.
%    \item This takes the form of applying an increment of $u_{inc} = \frac{U_{ref} - <u>}{\tau} \Delta t$. $\tau = $ \SI{21600}{s}.
%    \item The momentum flux, can be calculated using $\rho \mean{u' v'} = \rho \frac{u_{inc}}{\Delta t} \Delta z$.
%\end{itemize}

\newpage
\section{Results}

%\begin{itemize}
%   \item Table 1: List of experiments, S0, S4 etc. plus explanation of each one.
%   \item Figure 1: Two snapshots from S0/S4, with no organization and organization present. Something like $w$ at \SI{2500}{m}.
%\end{itemize}
%

% Figure \ref{fig:S0_S4_snapshot} shows \SI{15}{min} means of rainfall from two of the experiments, S0 and S4. 
%\begin{itemize}
%    \item \textbf{Figure \ref{fig:S0_S4_snapshot}:}
%    \item random (left) vs organized (right)
%    \item stoch process
%    \item linear organization - identified as a squall line
%    \begin{itemize}
%        \item size of cell
%        \item strength of cell
%        \item number of cells.
%    \end{itemize}
%    \item spatial distribution of clouds (there are areas of S4 with no clouds in them)
%\end{itemize}
%
In Figure \ref{fig:S0_S4_snapshot}, rainfall fields from two \SI{15}{minute} periods in the final day for the S0 and S4 experiments are shown. The difference in spatial organization between them is clear: S0 displays what looks to be a random or `popcorn' spatial distribution, whereas S4 is organized into a linear feature, perpendicular to the applied shear. We identify this linear feature as being a squall line. The squall line is propagating westward, with a speed of \SI{11.4}{m.s^{-1}}, or very close to the maximum wind speed. From the shear profile for S4, this implies that it is being controlled by a steering level of \SI{1}{km}. Also notable is the size and number of the convective cells in each experiment: the cells in S4 have larger areas, and there are fewer of them. The maximum grid-scale value of rainfall is larger for S0 at this time period, however there is more variation in the maximum grid-scale value of rainfall for S4. For example, over the final day, S0 has minimum and maximum values of rainfall of \SI{107}{}-\SI{217}{mm.hr^{-1}}, whereas S4 has \SI{0}{}-\SI{309}{mm.hr^{-1}}. However, the larger area of the convective cells in S4 suggest that although the grid-scale values of rainfall may be less in a given period, the rainfall per cloud will be higher.

Through examination of animations of the rainfall fields, we observe that in S4 the individual convective cells are longer lived than their S0 counterparts. The lifetime of a cell in S0 is limited by the typical convective cell lifecycle, where the convective downdraught caused by the evaporation of precipitation and precipitation drag acts to destroy conditions favourable for convection. However in the S4 experiment, there is a separation of updraughts and downdraughts, visible when looking at 2D vertical cross sections through a convective cell (not shown). We speculate that this separation increases the lifetime of the convective cells in the S4 experiment.
%However, in S4, the lifetime of a cell is in general shorter than lifetime of the squall line of which it is a part.

The spatial distribution also suggests another difference between the two experiments: that the variance of organized cases will be larger than the variance of non-organized cases. In S4, there is very little active convection in the some parts side of the domain. Whereas in S0, due to the random distribution of the convection, the convection is more evenly distributed over smaller spatial scales of the domain, and hence its variance will be less.

\begin{figure}[h]
    \centering
    \includegraphics[width=400px]{figs/{atmos.precip_plot.time_index3822_cropped}.png}
    \caption{Two rainfall fields from experiments S0 and S4, taken from the final day of each simulation, showing unorganized and organized behaviour. The field is a \SI{15}{minute} mean rainfall. Values of less than \SI{1e-4}{mm.hr^{-1}} are not shown. The maximum rainfall for each experiment for this period is: S0, \SI{163}{mm.hr^{-1}}; S4, \SI{109}{mm.hr^{-1}}.}
    \label{fig:S0_S4_snapshot}
\end{figure}

\subsection{Equilibrium profiles}
%\begin{itemize}
%    \item Figure 5: Difference in equilibrium states from the different experiments. 
%\end{itemize}
%

%\begin{itemize}
%    \item \textbf{Figure \ref{fig:thermodyn_profile}:}
%    \item \textbf{left:}
%    \begin{itemize}
%        \item surf theta up with S up. expl: higher u near surf for S up.
%        \item theta approx. const in BL
%        \item tropopause height up with S up
%        \item max cloud top height up with S up
%        \item theta in clouds gt theta domain mean. expl: latent heat release in the cloud that is precip'd out.
%        \item 4 $\frac{d \theta}{dz}$s: 1 km/LCL, 7-9 km, 12-14 km, - breaks
%    \end{itemize}
%    \item \textbf{right:}
%    \begin{itemize}
%        \item all show qcl maxima at around 2.5 km
%        \item [S2]S3-4 show secondary maxima at 5-7 km. expl: trailing stratiform regions
%        \item lower max mag. lower for S up. \todo{expl}, indicates a large number of clouds dying off at higher heights
%    \end{itemize}
%\end{itemize}

\begin{figure}[h]
    \centering
    \includegraphics[width=400px]{figs/{atmos.profile_plot.thermodynamic_profile}.png}
    \caption{Equilibrium profiles showing domain mean potential temperature (left), $\theta$, and the hydrometeors (right), averaged over the final day of each simulation. The potential temperature shows domain mean (solid) and in-cloud mean (dashed) potential temperature. The LCL is denoted by a cross, and the cloud top height by a dot. The hydrometeors shown are cloud liquid water (solid), graupel (dot), and ice (dash-dot). }
    % todo: indicate 0 degC level.
    \label{fig:thermodyn_profile}
\end{figure}

Profiles from each of the experiments of $\theta$ and hydrometeors are shown in Figure \ref{fig:thermodyn_profile}. For the $\theta$ profiles, two lines are plotted for each experiment, the solid line is the domain averaged $\theta$ profile, and the dashed line is the in-cloud $\theta$ profile. The markers at the bottom and top of the dashed lines therefore denote lifting condensation level (LCL) and cloud top height respectively. As can be seen from Figure \ref{fig:input_profiles}, in our experimental setup, surface winds increase with increasing shear profile. This explains why the surface value of $\theta$ is closer to \SI{300}{K}, the SST, for increasing shear profile, as a smaller temperature difference is needed to produce the required heat fluxes to balance the prescribed cooling. In all experiments, $\theta$ is approximately constant in the boundary layer. The LCL can be seen from the markers at the end of the dashed lines, this decreases slightly with increasing shear. 

%[Above the LCL, all experiments show approximately constant $\frac{d \theta}{dz}$ up to \SI{7}{}-\SI{9}{km}, with the height at which the gradient changes increasing with increasing shear \todo{why approx. const? Shouldn't it follow a moist-adiabat, i.e. be curved? Could be due to there being fewer clouds at higher heights? No obvious changes in qcl/#clds etc. at this height}. ]
%\todo{pick between descriptions:}
Above the LCL, all experiments approximately follow moist-adiabats \todo{plot tephigram}, although there appears to be a change of gradient at around \SI{7}{}-\SI{9}{km}, when the gradient decreases for all experiments. The high increase in $\theta$ with height can be attributed to the release of latent heat from the formation of the cloud liquid water and precipitation in the clouds.

The height of the tropopause can be seen from the location of the change in gradient of $\theta$, ranging from around \SI{12}{km} for S0, to around \SI{12.5}{km} for S2-4. The maximum cloud top height is above the tropopause in all experiments, and there is a general increase in this height with increasing shear, although clouds in S3 reach the greatest heights. This indicates that the strength of the individual convective cells increases with increasing shear, as the convective plumes then can overshoot the temperature inversion and their level of neutral buoyancy at the tropopause. 

%\todo{section on hydrometeors}
From the hydrometeors profile, we see that the vast majority of the hydrometeors are in the solid water phase. The top of the ice phase (dash-dot lines) is seen to correspond closely to the cloud top heights. The graupel is seen to descend lower into the troposphere than the ice. This could be because of its greater fall speed. However, little graupel survives to lower than around \SI{2.5}{km}, and at this height there is a maximum for all the experiments in cloud liquid water. This suggests that this maximum is caused by the melting of the graupel. This maximum is largest for S0, decreasing as the shear is increased. 

%\todo{weak!}
%From the profiles of cloud liquid water, there is a maximum in each of the experiments at approximately \SI{2.5}{km}. This maximum is greatest for the unorganized experiment, S0, and decreases with increasing shear. It indicates the presence of large numbers shallow cumulus clouds, which do not go on to form deep convective plumes. These shallow cumulus clouds therefore tend to be weaker than their deep counterparts.

%\todo{weak!}
%Experiments S2-4 all show a secondary maximum between \SI{5}{}-\SI{7}{km}. This can be explained by noting that this is the height at which a trailing stratiform region would form behind a squall line. This is not present for S0 or S1; this is a sign that there is a qualitative difference between the cloud fields between these two experiments and S2-4.

%\begin{itemize}
%    \item \textbf{Figure \ref{fig:mf_profile}:}
%    \item \textbf{left:}
%    \begin{itemize}
%        \item mf/cld up with S up
%        \item S2-4 all behave similarly
%        \item S0/1: mf/cld approx const, link to \cite{PC2008}
%        \item S2-4: mf/cld proportional to z, max at 10 km
%    \end{itemize}
%    \item \textbf{middle:}
%    \begin{itemize}
%        \item S0/1: number clds down with z up
%        \item S2-4: number of clouds approx const with height
%        \item max number clouds down with S up
%        \item S2-4: no secondary max at 5-7 km (cf. Fig \ref{fig:thermodyn_profile}). expl: presence of non-conv (stratiform) cld
%    \end{itemize}
%    \item \textbf{right:}
%    \begin{itemize}
%        \item S0: largest, max at 4 km
%        \item S2-4: similar, max at 6-8 km
%        \item \todo{say more!}
%    \end{itemize}
%\end{itemize}

\todo{reword this paragraph}
In the total convective mass flux, equal to the product of the mass flux per cloud and the number of clouds, S0 is seen to have the largest absolute value. This implies that the gradient $\frac{d \theta}{dz}$ must be least for this experiment, as the heating is due to subsidence, and $mf \times \frac{d \theta}{dz}$ is proportional to the heating. The other experiments all show more closely grouped values, with the organized experiments increasing in the upper troposphere, with maximum values between \SI{6}{}-\SI{8}{km}. 
%\todo{what more can I say about this?}

\subsection{Cloud field organization}

%\begin{itemize}
%    \item Figure \ref{fig:org}: Measure of organization demonstrating effectively the higher organization in the organized convection cases.
%\end{itemize}
The degree of clustering at a given spatial scale of each experiment can be investigated by calculating the normalized cloud number density as a function of the distance between the clouds (Figure \ref{fig:org}). This is essentially a histogram of how many clouds there are in a given annular region from every cloud, divided by the area of the annular region and normalized by the total number of clouds in the area \parencite{CC2006II}. A randomly distributed cloud field would have a value of 1 for all spatial scales (dashed line). This analysis tells us whether clouds form preferentially or are suppressed at a given spatial scale, dependent on whether the value at a given spatial scale is above or below the randomly distributed value of 1.

From Figure \ref{fig:org}, we can see that there is clustering at the smallest scales for all the experiments. For S0, we see a maximum value of around 3, and we see that the clustering only happens on a short spatial scale, less than around \SI{10}{km}. These findings agree closely with those of \cite{CC2006II}, who find a maximum value of the normalized cloud number density of around 2.9 for their \SI{2}{K.day^{-1}} experiment, and this value decreases to 1 by around \SI{10}{km} (their Figure 6). For S1, there is a similar signal, although the degree of clustering at short distances and the spatial scale over which the clustering are both marginally larger. S0 and S1 both appear to exhibit small amounts of repulsion at scales of \SI{10}{}-\SI{20}{km} and \SI{20}{}-\SI{40}{km} respectively.

\begin{figure}[H]
    \centering
    \includegraphics[width=400px]{figs/{atmos.org_plot.z1_combined_log}.png}
    \caption{The normalized cloud number density as a function of spatial scale for each experiment. A randomly distributed cloud field would have a value of 1 for all scales (blue dashed line).}
    \label{fig:org}
\end{figure}

For S2, there is more clustering at shorter scales, and the scale of clustering is larger. Now there is also distinct suppression of convection at scales between \SI{25}{km} and \SI{70}{km}. S3 and S4 both show very high levels of short range clustering, and this clustering now takes place on much larger spatial scales of up to \SI{50}{km}. There is also a degree of suppression between \SI{60}{km} and \SI{175}{km} for S3, and \SI{230}{km} for S4. Clustering appears to pick up again at \SI{250}{km}: this is due to the bi-periodic nature of the domain and is detecting the signal of the clustering of the linear convection with itself. These results are qualitatively consistent with the findings of \cite{CC2006II} (their Figure 6 (a)), although a direct comparison of numerical values is not possible as their sheared experiments were forced with \SI{8}{K.day^{-1}} cooling. However, here we go beyond the analysis that they performed, as the range of shear profiles allows us to estimate the transition between the organized experiments and the unorganized experiments. S0 and S1 both show qualitatively similar behaviour,  in that they show only a small amount of clustering at short scales, and have minimal suppression at larger scales, and so we label them the unorganized cases. S3 and S4 again show similar behaviour, in that they show large amount of clustering, a similar spatial scale for their clustering, and high values of suppression at larger spatial scales. We therefore label them the organized cases. S2 shows behaviour that is in between organized and unorganized, thus we call this the transition case.

% Doesn't make as much sense now org comes before other results.
%Overall the results from Figure \ref{fig:org} are broadly consistent with the other results in this study, in that they show a high degree of similarity for experiments S3-4. However, in the other results, S2 has appeared to be closer in nature to the organized experiments, whereas here it seems to show behaviour that is in between the organized and unorganized experiments. This would seem to indicate that shear profile that generated these data is in some sense ``on the cusp'', between stimulating organization and not.

\subsection{Mass flux}

\subsection{Mass flux profiles}

Figure \ref{fig:mf_profile} shows, from left to right, profiles of the number of clouds, the mass flux per cloud and the total convective mass flux. Clouds are diagnosed as set out in Section \ref{subsec:cloud_diag}. All three profiles shows that there is a difference between the two unorganized cases, S0 and S1, and the transition and organized cases, S2-4.

% \todo{weak!}
From the number of clouds in the leftmost figure in Figure \ref{fig:mf_profile}, there is a clear relationship: above \SI{1}{km}, as the shear increases, the number of clouds decreases. The clouds also start forming at a lower height for the strongly sheared experiments, which is consistent with the LCL heights in Figure \ref{fig:thermodyn_profile}. There is a distinction between the behaviour of the unorganized and organized cases. In the unorganized cases, the number of clouds decreases with increasing height, with a clearly defined maximum at \SI{1.5}{km}. However, in the organized cases, the number of clouds is approximately constant with height. S1, the transition case, is again somewhere in between, with a small decrease in the number of clouds with height.
%What is conspicuous by its absence is a secondary maximum in the number of clouds profile for the organized experiments. When compared to the cloud liquid water in Figure \ref{fig:thermodyn_profile}, this shows that the cloud diagnosis is effectively filtering out this region of stratiform cloud. This supports the assertion that the maximum in cloud liquid water is caused by a region of trailing stratiform cloud.

\begin{figure}[h]
    \centering
    \includegraphics[width=400px]{figs/{atmos.profile_plot.mf_profile}.png}
    \caption{Profiles of the number of clouds (left), the mass flux (MF) per cloud (centre) and the total mass flux (right) for each experiment. }
    \label{fig:mf_profile}
\end{figure}

The organized and transition cases all show an increase in mass flux per cloud as height increases, reaching maximum values at around \SI{10}{km}. This, coupled with the fact that the number of clouds is approximately constant for these cases, means that each convective updraught must be getting stronger. One possible explanation for this could be that the latent heat released in the updraught acts to strengthen the total mass flux in those plumes by increasing their buoyancy. In contrast, the mass flux per cloud in S0 is approximately constant over the troposphere, with values ranging from \SI{1e7}{kg.s^{-1}} just above the LCL to \SI{2.5e7}{kg.s^{-1}} by \SI{6}{km}. Speculatively, this suggests that there are two balancing effects in this case - there are fewer clouds at higher heights, and each cloud that makes it to a certain height will be stronger. Almost constant mass flux per cloud was also seen in \cite{PC2008}, although their study used cooling rates of \SI{4}{K.day^{-1}} and \SI{8}{K.day^{-1}} for their mass flux profiles. They find that the value for mass flux per cloud is insensitive to cooling rate however, and use the value \SI{2e7}{kg.s^{-1}} as their constant value, which is consistent with the range of values seen here. Mass flux per cloud is seen to increase as the shear increases. This could indicate that the shear has an inhibiting effect on convection, and so a larger and stronger cloud must form in order to overcome this inhibition.

%\subsection{Spatial variability}
%\begin{itemize}
%    \item Figure 2: Effects of spatial lengths on total mass flux (c.f. \cite{PC2008}, Fig. 1).
%    \item Figure 3: Effects of spatial lengths on PDF of mass flux per cloud cell (c.f. \cite{CC2006II}, Fig. 2).
%    \item N.B. perhaps combined into one figure.
%\end{itemize}

%\begin{itemize}
%    \item \textbf{Figure \ref{fig:mf_hists}:}
%    \item \textbf{upper:}
%    \begin{itemize}
%        \item this shows hist of $m$: $p(m) \times N_{cld}$
%        \item exponential decrease for S0, cf. \ref{eqn:pdf_mass_flux_per_cloud}
%        \item approx. exponential decrease for S1-4, but: fast drop at start? 2 slopes with break at \SI{1.25e8}{kg.s^{-1}}
%        \item S2-4: all similar
%    \end{itemize}
%    \item \textbf{lower:}
%    \begin{itemize}
%        \item this shows how much a given mf contributes to the total mf: $m \times p(m) \times N_{cld}$
%        \item as S up, max. contrib. comes from larger mf clds
%        \item S2-4 similar (again)
%    \end{itemize}
%\end{itemize}
%
\subsubsection{Variability of mass flux per cloud}
To test Equation \ref{eqn:pdf_mass_flux_per_cloud}, we present a histogram of the mass flux per cloud (Figure \ref{fig:mf_hists}, upper figure), at a height of \SI{2062}{m}. We find excellent agreement with this PDF for all the experiments, with an exponential decrease in the number of clouds with a given mass flux. Linear regressions have been fitted to the natural logarithm of each of the curves, using only the parts of the curve where there are more than 10 clouds These lines are seen to fit the data very well, with values of $r^2$ of between \SI{0.968}{} and \SI{0.996}{}. These results matches previous findings by \cite{CC2006II}, cf. their Figure 2. They find that the intercept between the curve and number of clouds occurs at a mass flux per cloud a value of \SI{1.2e8}{kg.s^{-1}}, we find using the extrapolation of our linear regression a value of \SI{1.79e8}{kg.s^{-1}}. The cloud diagnosis we are using is slightly different from the one they used: they used either a $w$ threshold or a cloud liquid water threshold; we use both together. They also performed their analysis at a different height: \SI{2.4}{km} instead of \SI{2062}{m}. 

\begin{figure}[H]
    \centering
    \includegraphics[width=400px]{figs/{atmos.mass_flux_plot.z1_both}.png}
    \caption{Top: histogram of the number of clouds with a given mass flux (solid line) for each experiment. A linear regression has been fitted to the natural logarithm of each curve (dashed line). Bottom: the histogram weighted by the mass flux, to give the total mass flux contribution of a given mass flux. }
    \label{fig:mf_hists}
\end{figure}

Additionally, to test the sensitivity of this analysis to the height at which it is carried out, we run the same analysis at \SI{1657}{m} and \SI{2518}{m}. We find that the analysis produces qualitatively similar results at these heights, although the exact number of clouds is different (not shown).

The organized and transition cases all show similar behaviour, with the gradient of each of these cases all having a similar value. However, the number of clouds with low mass fluxes decreases with increasing shear, consistent with Figure \ref{fig:mf_profile}, left. The maximum mass flux also increases with increasing shear, consistent with Figure \ref{fig:mf_profile}, centre.

Figure \ref{fig:mf_hists} also shows a measure of how much clouds with a given mass flux contribute to the total mass flux ($m \times p(m) \times N_{cld}$, where $m$ is the mass flux per cloud, $p(m)$ is its PDF and $N_{cld}$ is the total number of clouds). From this it is possible to see which types of clouds have the largest effect on the total mass flux. For S0, the clouds that have the largest effect are the ones with the lowest mass flux, i.e. there are a large number of clouds with a small mass flux. The maximum contribution increases with increasing shear; higher shear leads to larger clouds in terms of mass flux per cloud. Additionally, the experiments S1-4 are much broader, meaning there is a wider range of cloud sizes that contribute significantly to the overall convective mass flux.

%\begin{itemize}
%    \item \textbf{Figure \ref{fig:mf_spatial_hists}:}
%    \item \textbf{upper: S0: n=1, 2, 4}
%    \begin{itemize}
%        \item shows increase in variance with S incr
%        \item \todo{How to explain n=4 kink at low mf?}
%    \end{itemize}
%    \item \textbf{lower: n=1: S0-4}
%    \begin{itemize}
%        \item decrease in mean mf with s incr.
%        \item S2-4 similar
%        \item note, there are far fewer clds in the domain for higher S, but width of S0 vs e.g. S4 is similar
%        \item this means that variance of S2-4 is (relatively) lower (I think)
%    \end{itemize}
%\end{itemize}
%
\subsubsection{Variability of mass flux at different spatial scales}
We also want to know the variance of the different experiments at different spatial scales. To do this, we subdivide the domain into an equal number of square subregions, with lengths 1, 2 and 4 times smaller than the length of the domain. We then compute the histogram of the total convective mass flux within each subregion, and the results can be seen in Figure \ref{fig:mf_spatial_hists}. The analysis is again carried out at \SI{2062}{m}, although similar qualitative results are seen at \SI{1657}{m} and \SI{2518}{m} (not shown). The top figure shows the histogram for the S0 experiment over three different spatial scales. As was found in a previous study, the variance increases when the area of the subregions decreases \parencite{PC2008}. It should be noted that in their Figure 1, the prescribed cooling that they were using was far higher, \SI{16}{K.day^{-1}}, as opposed to \SI{2}{K.day^{-1}} used here. They therefore see far more clouds in their domain, which leads to a much narrower distribution. The qualitative results remain the same though, as they decrease the size of their subregions, they see an increase in variance.

\begin{figure}[h]
    \centering
    \includegraphics[width=400px]{figs/{atmos.mass_flux_spatial_scales_plot.both_z1}.png}
    \caption{Top: the rescaled frequency of mass fluxes for experiment S0, analysed over three different spatial scales, denoted by how much smaller they are than the length of the domain: 1, 2 and 4. Each of these has respectively subregions with an area of \SI{256}{} $\times$ \SI{256}{km^2}, \SI{128}{} $\times$ \SI{128}{km^2}, \SI{64}{} $\times$ \SI{64}{km^2}. Bottom: how the rescaled frequency distribution changes for the different experiments, using the full domain. Note that the solid blue lines are the same in the top and bottom figures. }
    \label{fig:mf_spatial_hists}
\end{figure}

In the lower figure in Figure \ref{fig:mf_spatial_hists}, we also compare the total convective mass flux across the whole domain for our different experiments. Note that the solid blue line is the same in both upper and lower figures. As the shear is increased, the total mass flux in the domain decreases, with the organized cases all showing similar distributions. The width of the distributions across all the experiments is broadly the same. 

%\todo{Don't fully understand this, can we discuss? Thought the variance increase would be bigger.} 
%This coupled with the fact that there are far fewer clouds in the high shear experiments, means that the relative variance \todo{not sure how to phrase this} is actually lower.

\newpage

% Discussion
% How results might lead to the modification of a stoch. p13n:
\section{Discussion}
%\begin{itemize}
%    \item How these results can inform the design of a stochastic parametrization.
%    \item Need to be able to characterize \textit{the shear profile}. Possibly through a bulk Richardson number.
%    \item Then link this to changes in $p(m)$, $p(M)$.
%    \item Then use this to modify steps 2, 3, outlined above for stochastic parametrization.
%    \item More speculatively, could also link cloud lifetimes to stochastic parametrization.
%\end{itemize}

%\todo{(sep from prev.?) Paragraph on spatial scale of org.}

%\todo{Paragraph on specific results - mass flux per cloud increases with org., number of clouds with low mf/cld decreases with org., number of clouds almost const with height for org'd conv.}

%\todo{Paragraph on variance of $p(M)$ - what goes here?}

Through using linear scalings of a shear profile, we have identified a range of behaviours, with unorganized cases at one end of the scale, and organized cases at the other. In between, the S2 experiment, although often exhibiting behaviour closer to the organized experiments, sits between these two extremes and so can be labelled the transition case. For example, in Figure \ref{fig:org}, the distance over which it shows enhanced clustering is not as great as the organized cases. However, in the other figures S2 is seen to lie close to the organized cases. Although this is a somewhat arbitrary way of dividing the experiments, it may prove useful in the future to have a means for determining when to use a qualitatively different convection parametrization due to organization, instead of treating the different shear profiles as lying on a spectrum of behaviour. This would require conclusive evidence that there was qualitative difference between the cases though, and a clear determination of where and how the transition between unorganized convection happens. This could form the basis of future work.

% How results can be used.
These results represent a first step towards trying to include some of the effects of spatial organization into a stochastic convection parametrization scheme. We have shown that there is a clear dependence of the distribution of mass flux per cloud on the strength of shear profile, and so this provides a route into modifying an existing stochastic parametrization scheme by changing the formulation of this PDF. Likewise, the distribution of total convective mass flux over a given spatial scale shows a clear relationship with the strength of shear profile. This again leads to a natural means for modifying a stochastic parametrization scheme. With reference to the outline of how a stochastic parametrization scheme works, modifying these PDFs would amount to changing the behaviour of the parametrization scheme in steps 2 and 3.

% Limitations.
The work here is limited in three key ways. First, although it has potential for modifying a convection parametrization scheme, it only provides part of the answer as to how this might be done. This study looks at how key PDFs change with shear, and thus organization, in simulations; it provides no way of characterizing the shear profile in a way that could be used in a parametrization scheme. In this study we control the shear profile; however, when running as part of a parametrization scheme, the shear profile will be determined by the dynamics of the model, and will not in general match one of the shear profiles studied here. It will therefore be necessary to have a way of characterizing shear profiles generated by the model dynamics, and matching these to changes in the PDFs of mass flux. One possible way of characterizing the shear profiles would be through a bulk Richardson number, which provides a measure of the likelihood of turbulence by dividing the static stability by the shear over a given layer.

Second, we considered only deep shear profiles, we did not look at low level shear only experiments, where the shear is primarily in the boundary layer. Low level shear has been hypothesized to be fundamental in the role of squall line formation, due to the interaction of the convective cold pool with the environmental air with an opposite vorticity \parencite{RKW1988}. CRM studies have looked at this phenomenon in detail \parencite{RE2001, TMM1982}, although they have not focused on the statistics of the mass flux variability. We have performed experiments similar to those in \cite{RE2001}, however we did not see similar qualitative results, in the form of shear parallel squall lines. One possible reason for this is that the resolution that we are running with is too coarse for our CRM to resolve the dynamical processes that are responsible for this type of behaviour, although we are using the same resolution as in \cite{RE2001}. Stimulating squall lines with this type of profile would be desirable, as it provides a simpler profile, controlled by fewer parameters, and therefore makes experiments that search over the parameter space easier to design.

Third, stochastic convection parametrization schemes such as \cite{PC2008} also require an estimate of the lifetime of the convective cells. Modifications to the lifetime of clouds has been tried in a stochastic shallow cumulus convection scheme \parencite{sakradzija2016stochastic} with some success, and so could provide benefits for a stochastic deep convection scheme as well. From our results, there are reasons to believe that this increases as the shear increases. However, we have only noted this qualitatively, if we were to use a cloud tracking procedure we could be quantitative about this relationship. This would again lead to a possible modification to stochastic convection parametrization schemes.

%\todo{Could add a section on convective momentum transports here too.}

% Conclusion
\section{Conclusions}
%\begin{itemize}
%    \item We have demonstrated that shear can control the degree of organization in RCE cases.
%    \item That there is a qualitative difference that manifests itself across a wide range of metrics between organized and unorganized conv.
%    \item That this leads to clear differences between the size, strength and number of convective cells.
%    \item These differences can be characterized by e.g. $p(m)$, $p(M)$ .
%    \item It has hinted at differences in lifecyles between the organized and unorganized expts.
%\end{itemize}
%

% degree of org., spatial scale of org.
In this study we have demonstrated that the degree of organization in an idealized atmospheric model can be controlled by applying linearly scaled deep shear profiles. When there is no shear, the convection is unorganized. As the shear is increased, the degree of organization increases. The spatial scale of the organization was calculated. We found that the organized experiments exhibit preferential organization on scales up to \SI{50}{km}, and inhibition of convection beyond this length scale.

% shear up -> mf/cld up, #clds down. spec. cloud lifetime up with shear up.
Increasing shear has also been seen to increase the mass flux per cloud, and decrease the number of clouds in the domain. There is evidence that the lifetime of the convective cells increases with increasing shear as well.

% Qual diff between org'd/unorg'd expts.
The organized and unorganized cases show qualitatively different behaviour for several metrics. The organized cases show less variation of cloud number with height, the mass flux per cloud increases with height and there is a maximum in the total convective mass flux at higher levels. The distribution of number of clouds with mass flux also shows that there are far fewer clouds with lower mass flux in the organized cases, and the maximum mass flux per cloud is around 2 times higher for the organized cases. The mass flux which contributes the most to the total convective mass flux is also higher for the organized cases. In general, the transition case's behaviour is close to the organized cases, although in terms of the spatial scales of the organization it shows some significant difference.

% Variance of mf with spat scales.
The variance of mass flux with different spatial scales is seen to increase as the scale of analysis is decreased. The variance remains comparable when the same spatial scale is analysed for the different experiments though.

%\newpage
%\subsection*{Main papers (*:key papers)}
%\begin{itemize}
%    \item * \cite{CC2006I}
%    \item * \cite{CC2006II}
%    \item * \cite{RE2001}
%    \item * \cite{RKW1988}
%    \item * \cite{PC2008}
%    \item \cite{birch2014scale}
%    \item \cite{cohen2004response}
%    \item \cite{gregory1997parametrization}
%    \item \cite{houze1977structure}
%    \item \cite{kershaw1997parametrization}
%    %\item \cite{parker2007simulated}
%    \item \cite{robe1996moist}
%    \item \cite{sakradzija2016stochastic}
%    \item \cite{sengupta1990cumulus}
%    \item \cite{TMM1982}
%\end{itemize}
%
%
\printbibliography[title={References}]

\end{document}
