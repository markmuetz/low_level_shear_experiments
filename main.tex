\documentclass[11pt,a4paper]{article}

\usepackage[utf8]{inputenc}
\usepackage{siunitx}
\usepackage{geometry}
\newgeometry{margin=2.0cm}

\usepackage[backend=biber,style=authoryear,sorting=nyt,dashed=false]{biblatex}
\renewcommand*{\nameyeardelim}{\addcomma\space}
\addbibresource{references/references.bib} % note the .bib is required

\newcommand*\mean[1]{\overline{#1}}

\title{Effects of Low Level Shear on Cloud Field Variability and Organization }
\author{Mark Muetzelfeldt}
\date{April 2017}

\begin{document}

\maketitle

\begin{itemize}
    \item \cite{birch2014scale}
    \item \cite{cohen2004response}
    \item \cite{CC2006I}
    \item \cite{CC2006II}
    \item \cite{gregory1997parametrization}
    \item \cite{houze1977structure}
    \item \cite{kershaw1997parametrization}
    \item \cite{parker2007simulated}
    \item \cite{robe1996moist}
    \item \cite{RE2001}
    \item \cite{PC2008}
    \item \cite{sakradzija2016stochastic}
    \item \cite{sengupta1990cumulus}
    \item \cite{TMM1982}
\end{itemize}

\section{Introduction}

\begin{itemize}
    \item Justification for using RCE - links to QE.
    \item Motivation for how this work fits into larger picture of stochastic parametrization. Particularly the variational analysis.
    \item Set out experimental design: choice of shear profiles and explanation of why these produce organization.
    \item Mention: organization and momentum flux.
\end{itemize}

\section{Model and experimental design}

\subsection{Model setup}

\begin{itemize}
    \item Using an idealized version of the Met Office Unified Model (version 10.7).
    \item 2 spatial resolutions used: \SI{2}{km} and \SI{500}{m}. The first allows easier comparison between this and previous studies, the second is better able to capture convection. Running with $\Delta t$ of \SI{30}{s} and \SI{10}{s} respectively.
    \item Running over a domain of 256$\times$256 \SI{}{km^2}, for a duration of \SI{18}{days}. This should be large enough to cover the spatial extent of the squall lines, and long enough to both reach equilibrium and provide a long enough time to average over many squall lines' lifetimes.
    \item Forced with prescribed cooling of \SI{1.5}{K.day^{-1}} - typical of radiative cooling rates over the tropics.
    \item Using a 3D Smagorinsky Turbulence scheme, Smith 1990 MP, no convection scheme. Includes the OCF MC scheme.
    \item Applying a relaxation back to a neutral profile above \SI{12.5}{km}.
\end{itemize}

\subsection{Shear profiles}

\begin{itemize}
    \item Profiles are applied by relaxing back to a mean wind across the domain.
    \item This takes the form of applying an increment of $u_{inc} = \frac{U_{ref} - <u>}{\tau} \Delta t$. $\tau = $ \SI{21600}{s}.
    \item The momentum flux, can be calculated using $\rho \mean{u' v'} = \rho u_{inc}  \Delta t \Delta z$.
\end{itemize}

\section{Results}
\begin{itemize}
   \item Table 1: List of experiments, I0, I5 etc. plus explanation of each one.

   \item Figure 1: Two snapshots from I0/I5, with no organization and organization present. Something like $w$ at \SI{2500}{m}.
\end{itemize}

\subsection{Spatial variability}
\begin{itemize}
    \item Figure 2: Effects of spatial lengths on mass-flux.
    \item Figure 3: Effects of spatial lengths on PDF of mass-flux per cloud cell.
    \item N.B. perhaps combined into one figure.
\end{itemize}

\subsection{Cloud field organization}
\begin{itemize}
    \item Figure 4: Measure of organization demonstrating effectively the higher organization in the organized convection cases.
\end{itemize}

\subsection{Equilibrium profiles}
\begin{itemize}
    \item (Will this be kept?) Figure 5: Difference in equilibrium states from the different experiments. 
\end{itemize}

\subsection{Momentum transports}
\begin{itemize}
    \item Figure 6: Momentum flux profiles for different experiments. 
\end{itemize}
\section{Conclusions}

\newpage
\printbibliography[title={References}]


\end{document}
