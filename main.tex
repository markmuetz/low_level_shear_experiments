\documentclass[11pt,a4paper]{article}

\usepackage[pdftex]{color,graphicx}
\usepackage[utf8]{inputenc}
\usepackage{siunitx}
\usepackage{geometry}
\newgeometry{margin=2.5cm}

\usepackage[backend=biber,style=authoryear,sorting=nyt,dashed=false]{biblatex}
\renewcommand*{\nameyeardelim}{\addcomma\space}
\addbibresource{references/references.bib} % note the .bib is required

\newcommand*\mean[1]{\overline{#1}}
\newcommand\todo[1]{\textbf{TODO: #1}}

\title{Effects of Vertical Shear on Cloud Field Variability and Organization }
\author{Mark Muetzelfeldt, Robert Plant and  Peter Clark}
\date{June 2017}

% General notes:
% Be quantitative
% Do not make woolly comparisons
% Think about (sampling) errors 
% Be clear about speculation
% If speculating, ask yourself why you haven't made a better case for something
% Think about what evidence you need to support a conclusion - e.g. difference between distn integral vs 99th percentile


% Glossary:
% QE: Quasi-Equilibrium
% PDF: Probability Density Function
% CRM: Cloud Resolving Model
% UM: Unified Model
% SISL: Semi-Implicit, Semi-Lagrangian
% LCL: lifting condensation level

\begin{document}

\maketitle
\section{Introduction}

%\begin{itemize}
%    \item Justification for using RCE - links to QE.
%    \item Motivation for how this work fits into larger picture of stochastic parametrization. Particularly the variability analysis.
%    \item Mention: organization and momentum flux.
%\end{itemize}
Most convection parametrization schemes in use are deterministic - for a given grid-column state they will produce a single specific amount of convective heating, moistening and perhaps momentum transport. This can be justified through the Quasi-Equilibrium (QE) assumption of \cite{arakawa1974interaction}, in which it is assumed that the convection is in local equilibrium with the grid-scale forcing because each grid-column contains many convective plumes. As model resolution increases, this assumption will break down, as each grid-column contains fewer and fewer convective plumes. One way of modelling the parametrized convection in these circumstances is by using a statistical physics stochastic parametrization scheme \parencite{berner2017stochastic} - the magnitude of the subgrid convection is now determined by sampling from a Probability Density Function (PDF) instead of being uniquely determined by the grid-scale state.

\todo{mention sampling error}
Statistical stochastic parametrization schemes require a characterization of the subgrid variability of the convection. This can be done by using Cloud Resolving Models (CRMs) to verify theoretical relationships of the statistical properties of the convection. In this study, we look at how the organization of deep convection affects the variability of the convection. In particular, we look at how squall line like features generated by deep vertical shear in the troposphere modify the distribution of convective mass flux.

%\todo{theoretical relationships: mean mass flux per cloud p(m), total mass flux: p(M), CC2006i (6) and (14). I'd like to introduce these here, along with explanation that these are what I would like to test, with the tentative hypothesis that org will affect the 2nd by increasing the variability at a given scale. What should I say about the 1st? On looking at the data, I think that increased lifetime is leading to larger overall clouds that persist longer. This appears as a decrease in lower mf clouds, and increase in larger, still described by an exponential decrease. Should this go after next para? }

A theoretical relationship for the PDF of mass flux per cloud, $m$, was derived in \cite{CC2006I}. Under assumptions of weakly interacting clouds in statistical equilibrium, they showed that the PDF of $m$ decreased exponentially with increasing $m$:
\begin{equation}
    p(m) = \frac{1}{\langle m \rangle} \exp \left(\frac{-m}{\langle m \rangle} \right),
    \label{eqn:pdf_mass_flux_per_cloud}
\end{equation}
where angle brackets denote an ensemble average. They again use the weakly interacting clouds assumption to say that the number of clouds in a given region will be an independent variable, and can therefore be represented by a Poisson distribution. This, along with Equation \ref{eqn:pdf_mass_flux_per_cloud}, leads to a PDF for the total convective mass flux in a given region, $M$:
\begin{equation}
    p(M) = \sqrt{\frac{\langle N \rangle}{\langle m \rangle}} \exp \left( -\langle N \rangle M^{-\frac{1}{2}} \right) \exp \left( \frac{-M}{\langle m \rangle} \right)  I_1\left(2 \sqrt{\frac{\langle N \rangle}{\langle m \rangle} M}\right).
    \label{eqn:pdf_mass_flux}
\end{equation}
Here, $\langle N \rangle$ denotes the ensemble average number of clouds in a given region. $I_1(x)$ is the modified Besself function of order 1.

When these theoretical relationships are compared to distributions taken from numerical simulations, they are found to hold up well \parencite{CC2006II}. It is however necessary to correct for the finite size of clouds when comparing Equation \ref{eqn:pdf_mass_flux} to numerically modelled distributions. They also apply deep shear across the domain, and find that the the shear ``has only a small effect on the mass flux statistics'', even though the shear causes organization of the cloud field. 
% \todo{tenses} OK? - shifted all to present tense
It is this finding that we investigate, through the use of systematically varying linear shear profiles. In particular, we test how the mass flux per cloud is affected by increasing the cloud field organization by increasing the shear. Also, we look at how the distribution of $M$ changes as the organization increases.

CRMs can exhibit different modes of organization. For example, when run with interactive radiation, CRMs can display convective self-aggregation \parencite{wing2017convective}. 
This is where convection develops in a large subregion of the domain, often related to the domain size of the model, whilst outside of this region compensating subsidence occurs. Many mechanisms for this have been found, and may be different from model to model. 
However, longwave radiation feedback has been found to be essential for its formation \parencite{wing2017convective}. As we wish to isolate the organization stimulated by shear, we use a uniform prescribed cooling profile over our domain, thus removing longwave radiation feedbacks and ensuring that convective self-aggregation will not occur.

\todo{Vertical shear and its role in org. Difference between LLS/Deep Shear, evidence for both. Set out the fact I will be using deep. Vertical shear as a means of mom transport. Inclusion of this in p13n.}

The simulations carried out are run to statistical equilibrium, in the sense that the model state taken over a large enough domain and time period is statistically indistinguishable from another such period. Thus the model as a whole is [representative?] of the QE assumption. However, at smaller spatial or temporal scales, there will be significant departures from this mean state. It is precisely these fluctuations, and their relative frequencies, that we wish to characterize.

\todo{Stoch. p13n/QES: talk about why QES/RCE is a good way of running an RCE expt for use in a stoch. p13n. How you can use the results, the inverse model etc.}

Convective mass flux plays an important role in many convection parametrization schemes. The magnitude of the mass flux is typically determined through the use of a closure based on the grid-column state of the model. A popular choice of closure is the Convective Available Potential Energy (CAPE) closure assumption, whereby a percentage of the CAPE is assumed to be consumed by the convection in a given timestep. This can be used to set the mass flux by iterating towards the required value necessary to remove the desired amount of CAPE. The magnitude of the mass flux then determines the profile of heating and moistening of the atmosphere that will happen because of the unresolved convection. This is calculated through the use of a cloud model.

\todo{How MF distn can be fed into a statistical stoch p13n.}
Understanding how the convective mass flux fluctuates around a mean value can therefore be used to build a statistical physics stochastic parametrization scheme. For deep convection, this idea was used by \cite{PC2008}, and for shallow convection this has been done by \cite{sakradzija2016stochastic}. 

%The theoretical relationships allow us the characterize how mass flux will vary. Mass flux is a key parameter in convection parametrization schemes, as it value is used to determine the behaviour of a cloud model. This will typically lead to a moistening of the upper troposphere through detrainment from the cloud, as well as a heating of the troposphere, primarily through the compensating subsidence. The magnitude of the mass flux is determined through the use of a closure based on the grid-column state of the model. A popular choice of closure is the Convective Available Potential Energy (CAPE) closure assumption, whereby a percentage of the CAPE is assumed to [be consumed] by the convection in a given timestep. This can be used to set the mass flux by iterating towards the required value or by theoretical arguments. The magnitude of the mass flux then determines how much heating and moistening of the atmosphere will happen because of the unresolved convection, through the use of a cloud model.

\section{Model and experimental design}

\subsection{Model configuration}

%\begin{itemize}
%    \item Using an idealized version of the Met Office Unified Model (version 10.7).
%    \item Use resolutions of \SI{1}{km}.
%    \item Running over a domain of 256$\times$256 \SI{}{km^2}, for a duration of \SI{20}{days}. This should be large enough to cover the spatial extent of the squall lines, and long enough to both reach equilibrium and provide a long enough time to average over many squall lines' lifetimes.
%    \item Forced with prescribed cooling of \SI{2}{K.day^{-1}} - typical of radiative cooling rates over the tropics.
%    \item Using a 3D Smagorinsky Turbulence scheme, no convection scheme. Includes the OCF MC scheme.
%    \item Applying a relaxation back to a neutral profile above \SI{12.5}{km} (or \SI{15}{km}?).
%    \item Set out experimental design: choice of shear profiles and explanation of why these produce organization, linking back to previous work, e.g. \cite{RKW1988}.
%\end{itemize}

% Model, resolution, domain.
The model used is the UK Met Office Unified Model, version 10.7 \parencite{walters2017met}, run in its idealized configuration. It is a fully compressible, non-hydrostatic model, and its dynamics are modelled using a Semi-Implicit, Semi-Lagrangian (SISL) advection scheme. It is run using Cartesian coordinates, over a bi-periodic domain of 256$\times$256 \SI{}{km^2}, using a horizontal resolution of \SI{2}{km} and a \SI{30}{s} timestep. It uses a variable vertical resolution taken from the Large Eddy Model \parencite{todocite}, starting at \SI{63.22}{m} at the surface, increasing to \SI{250}{m} by \SI{2.5}{km}. The resolution increases again above \SI{15}{km}, increasing to \SI{1}{km} by \SI{25}{km} where at remains up to the top of the model at \SI{40}{km}. \todo{Add vert res profile to Figure \ref{fig:uv_profile}}
The experiments are run for \SI{20}{days}, enough time for the simulations to reach equilibrium and for sufficient statistics about the model state to be gathered. 

No convection parametrization scheme is used, instead the convection is explicitly modelled by the dynamics and the subgrid parametrization schemes of the model that are in use. The model uses a 3D Smagorinsky subgrid turbulence mixing scheme \parencite{todocite}. It includes prognostic ice and graupel in its microphysics, as well as a two moment moist microphysics scheme \parencite{todocite}. \todo{BL}. The surface is treated as ocean, with a fixed temperature of \SI{300}{K} and a friction roughness length of \SI{0.0002}{m}. The surface heat and moisture fluxes are modelled using bulk aerodynamic formulae, using the same roughness length of \SI{0.0002}{m}.

All simulations are run with a Coriolis parameter, $f$, equal to 0. They are run without geostrophic forcing.

\subsection{Moisture and energy balance}

The SISL advection scheme allows for a large timestep without becoming unstable. Such schemes are not conservative; there is no guarantee that there will be the same total quantity of a tracer from one timestep to the next. In preliminary tests, we found that this had a large impact on the total water content in the model. The non-conservation of moisture leads to precipitation rates being three times higher than what would be expected through comparison with evaporation rates. To ameliorate this, we used the Optimal Conservative Filter (OCF) moisture conservation scheme \parencite{zerroukat2015monotonic}, developed previously for this purpose. This scheme is similar to the Priestley scheme \parencite{priestley1993quasi}, in that it compares the total water content at two consecutive timesteps and ensures conservation by adding or removing water vapour as necessary from grid-cells. It does this in a way that is consistent with the uncertainty caused by using interpolation in the SISL scheme, using linear and higher order interpolation as bounds in which it can modify the water vapour in a grid-cell. The OCF scheme differs from the Priestley scheme only in the way it redistributes moisture: it attempts to do it in a way that is locally smooth \todo{check this}. With the OCF scheme in use, moisture conservation is guaranteed and once the model has reached equilibrium, precipitation balances evaporation as it should.

In a similar manner to moisture, the SISL scheme can also lead to a non-conservation of energy. No suitable scheme for energy conservation is available in the UM, and so the effects of this energy non-conservation must be characterized and taken into account. This manifests itself as an imbalance between the prescribed cooling energy loss, and the energy input into the model through the surface latent and sensible heat fluxes. Essentially, energy is being supplied by the dynamics of the model, which means that the surface fluxes are less than what would be expected to balance the prescribed cooling by around \SI{50}{W.m^{-2}}.
For the purposes of this study, in which we wish to study the dynamical interactions between clouds, this is not an issue. This is supported by the fact that the imbalance mainly affects the stratosphere, i.e. it is far from the area of interest in this study. However, if a detailed energy budget were the focus of the investigation, this would become of prime importance and it would be necessary to either account for the imbalance rigorously, or to apply a similar scheme as the OCF scheme described above to the energy balance. 

\subsection{Initial and cooling profiles}

Each simulation is started with an initial moisture and potential profile taken from the final state of a previous reference run. \todo{what can I say about this other than it comes from an LEM simulation?}. Simulations are allowed to spin-up for \SI{5}{days} while the surface fluxes come into balance with the energy and precipitation losses from the model. Output from the model is only analysed after the initial spin-up period.

A prescribed cooling rate of \SI{2}{K.day^{-1}} is applied from the surface to \SI{200}{hPa}, decreasing linearly to \SI{0}{K.day^{-1}} by \SI{100}{hPa}. This cooling rate is comparable to typical radiative cooling rates over the tropics. Larger cooling rates representing large-scale convergence and forced ascent, ranging from \SI{4}{K.day^{-1}} to \SI{16}{K.day^{-1}}, were also tested. However, when the model was forced with these rates, it developed a large-scale, domain wide oscillation, thought to be a convectively coupled gravity wave with period equal to the largest scale available to the disturbance - the length of the domain. These oscillations are a form of organization, and so affect the fluctuations that we are trying to measure. They are also an artefact of the modelling process due to the bi-periodic nature of the boundary conditions, and so are little more than a curiosity and an annoyance. We therefore decided to run with smaller cooling rates that did not stimulate this mode of organization.
\todo{add cooling profile to Figure \ref{fig:uv_profile}}

\subsection{Shear profiles and experiments}

The shear profile that we choose to impose was originally derived from the Marshall Islands experiments \parencite{yanai1973determination}. It has subsequently been used to stimulate mesoscale organization in the form of linear features resembling tropical squall lines, in both 2D \parencite{tompkins2000impact} and 3D \parencite{grabowski1996long, CC2006II} experiments. It is applied by relaxing back to a mean wind across the domain, by applying a single increment to all grid-cells at a given height level in the domain. In the east-west direction, the increment at each height are given by: $u_{inc} = \frac{U_{ref} - \langle u \rangle}{\tau} \Delta t$, where $U_{ref}$ is our reference profile, $\langle u \rangle$ is the domain mean $u$ wind, $\tau = $ \SI{21600}{s}, and $\Delta t$ is the model timestep, or \SI{30}{s}. 

\todo{S0-S5 - linear scalings of the basic profile.}

The reference profiles, along with the mean wind from the model over the final day of the simulation for each of the experiments, is given in Figure \ref{fig:uv_profile}.

\begin{figure}[htp!]
    \centering
    \includegraphics{figs/{atmos.profile_plot.uv_profile}.png}
    \caption{\todo{add vert res, cooling profiles} The reference profile, and the mean wind over the full domain averaged over the final day of each simulation, S0-S5. }
    \label{fig:uv_profile}
\end{figure}

\subsection{Cloud diagnosis}
\label{subsec:cloud_diag}

\todo{description of how clouds defined for analysis, plus some mention of sensitivity to thresholds.}
%\begin{itemize}
%    \item Profiles are applied by relaxing back to a mean wind across the domain.
%    \item This takes the form of applying an increment of $u_{inc} = \frac{U_{ref} - <u>}{\tau} \Delta t$. $\tau = $ \SI{21600}{s}.
%    \item The momentum flux, can be calculated using $\rho \mean{u' v'} = \rho \frac{u_{inc}}{\Delta t} \Delta z$.
%\end{itemize}

\section{Results}

%\begin{itemize}
%   \item Table 1: List of experiments, S0, S5 etc. plus explanation of each one.
%   \item Figure 1: Two snapshots from S0/S5, with no organization and organization present. Something like $w$ at \SI{2500}{m}.
%\end{itemize}
%

% Figure \ref{fig:S0_S4_snapshot} shows \SI{15}{min} means of rainfall from two of the experiments, S0 and S4. 
%\begin{itemize}
%    \item \textbf{Figure \ref{fig:S0_S4_snapshot}:}
%    \item random (left) vs organized (right)
%    \item stoch process
%    \item linear organization - identified as a squall line
%    \begin{itemize}
%        \item size of cell
%        \item strength of cell
%        \item number of cells.
%    \end{itemize}
%    \item spatial distribution of clouds (there are areas of S4 with no clouds in them)
%\end{itemize}
%
In Figure \ref{fig:S0_S4_snapshot}, rainfall from two \SI{15}{minute} periods in the final day for the S0 and S4 experiments are shown. The difference in spatial organization between them is clear: S0 displays what looks to be random or `popcorn' spatial distribution, whereas S4 is organized into a linear feature, perpendicular to the applied shear. We identify this linear feature as being a squall line. This squall line is propagating westward, with a speed of \todo{what speed}. From the shear profile for S4, this implies that it is being controlled by a steering level of \todo{what level?}. Also notable is the size, strength and number of the convective cells in each experiment: the cells in S4 are larger, have higher maximum rainfall rates, and there are fewer of them. 

Through examination of high temporal resolution (\SI{5}{minute}) snapshots, we observe that in S4 the individual cells are not continually forming and dissipating, rather they are indeed long-lived structures that persist for \todo{how long?}. \todo{link back to what I have not yet said in intro - about the nature of the organization formed by deep shear in the atm}. Looking at animations of this field, we observe that in S4 the lifetime of an individual convective cell is longer than in S0. However, the lifetime of a cell is in general shorter than lifetime of the squall line of which it is a part.

The spatial distribution also suggests another difference between the two experiments: that the variance of organized cases will be larger than the variance of non-organized cases. In S4, there is very little active convection in the right-hand side of the domain. Whereas in S0, due to the random distribution of the convection, the convection is more evenly distributed over smaller spatial scales of the domain \todo{clunky phrasing}.

\begin{figure}[htp!]
    \centering
    \includegraphics[width=400px]{figs/{atmos.precip_plot.time_index1870_cropped}.png}
    \caption{\todo{rename scale rainfall - it doesn't include snow etc.}, \todo{need to get max rainfall for each expt.}}
    \label{fig:S0_S4_snapshot}
\end{figure}

\subsection{Equilibrium profiles}
%\begin{itemize}
%    \item Figure 5: Difference in equilibrium states from the different experiments. 
%\end{itemize}
%

%\begin{itemize}
%    \item \textbf{Figure \ref{fig:thermodyn_profile}:}
%    \item \textbf{left:}
%    \begin{itemize}
%        \item surf theta up with S up. expl: higher u near surf for S up.
%        \item theta approx. const in BL
%        \item tropopause height up with S up
%        \item max cloud top height up with S up
%        \item theta in clouds gt theta domain mean. expl: latent heat release in the cloud that is precip'd out.
%        \item 4 $\frac{d \theta}{dz}$s: 1 km/LCL, 7-9 km, 12-14 km, - breaks
%    \end{itemize}
%    \item \textbf{right:}
%    \begin{itemize}
%        \item all show qcl maxima at around 2.5 km
%        \item [S2]S3-5 show secondary maxima at 5-7 km. expl: trailing stratiform regions
%        \item lower max mag. lower for S up. \todo{expl}, indicates a large number of clouds dying off at higher heights
%    \end{itemize}
%\end{itemize}

\begin{figure}[htp!]
    \centering
    \includegraphics[width=400px]{figs/{atmos.profile_plot.thermodynamic_profile}.png}
    \caption{\todo{add markers at both ends of the dashed lines}}
    \label{fig:thermodyn_profile}
\end{figure}

Profiles from each of the experiments of $\theta$ and cloud liquid water are shown in Figure \ref{fig:thermodyn_profile}. For the $\theta$ profiles, two lines are plotted for each experiment, the solid line is the domain averaged $\theta$ profile, and the dashed line is the in-cloud $\theta$ profile. The markers at the bottom and top of the dashed lines therefore denote lifting condensation level (LCL) and clouds top height respectively. As can be seen from Figure \ref{fig:uv_profile}, in our experimental setup, surface winds increase with increasing shear profile. This explains why the surface value of $\theta$ is closer to \SI{300}{K}, the SST, for increasing shear profile, as a smaller temperature difference is needed to produce the required heat fluxes to balance the prescribed cooling. In all experiments, $\theta$ is approximately constant in the boundary layer. The LCL can be seen from the markers at the end of the dashed lines, this decreases slightly with increasing shear \todo{why?}. 

\todo{pick between descriptions:}
[Above the LCL, all experiments show approximately constant $\frac{d \theta}{dz}$ up to \SI{7}{}-\SI{9}{km}, with the height at which the gradient changes increasing with increasing shear \todo{why approx. const? Shouldn't it follow a moist-adiabat, i.e. be curved? Could be due to there being fewer clouds at higher heights? No obvious changes in qcl/#clds etc. at this height}. ]
[Above the LCL, all experiments approximately follow moist-adiabats \todo{I can't justify this except by guessing/ by eye at the mo}, although there appears to be a change of gradient at around \SI{7}{}-\SI{9}{km}, when the gradient decreases for all experiments \todo{why?}.] The increase in $\theta$ with height can be attributed to the release of latent heat from the formation of the precipitation in the clouds.

The height of the tropopause can be seen from the location of the change in gradient of $\theta$, ranging from around \SI{12}{km} for S0, to around \SI{12.5}{km} for S2-5. The maximum cloud top height is above the tropopause in all experiments, and there is a monotonic increase in this height with increasing shear. This indicates that the strength of the convection increases with increasing shear, as the convective plumes then can overshoot the temperature inversion and their level of neutral buoyancy at the tropopause.

From the profiles of cloud liquid water, there is a maximum in each of the experiments at approximately \SI{2.5}{km}. This maximum is greatest for the unorganized experiment, S0, and decreases with increasing shear. It indicates the presence of large numbers shallow cumulus clouds, which do not go on to form deep convective plumes. These shallow cumulus clouds therefore tend to be weaker than their deep counterparts.

Experiments S2-5 all show a secondary maximum between \SI{5}{}-\SI{7}{km}. This can be explained by noting that this is the height at which a trailing stratiform region would form behind a squall line. This is not present for S0 or S1; this is a sign that there is a qualitative difference between the cloud fields between these two experiments and S2-5.

%\begin{itemize}
%    \item \textbf{Figure \ref{fig:mf_profile}:}
%    \item \textbf{left:}
%    \begin{itemize}
%        \item mf/cld up with S up
%        \item S2-5 all behave similarly
%        \item S0/1: mf/cld approx const, link to \cite{PC2008}
%        \item S2-5: mf/cld proportional to z, max at 10 km
%    \end{itemize}
%    \item \textbf{middle:}
%    \begin{itemize}
%        \item S0/1: number clds down with z up
%        \item S2-5: number of clouds approx const with height
%        \item max number clouds down with S up
%        \item S2-5: no secondary max at 5-7 km (cf. Fig \ref{fig:thermodyn_profile}). expl: presence of non-conv (stratiform) cld
%    \end{itemize}
%    \item \textbf{right:}
%    \begin{itemize}
%        \item S0: largest, max at 4 km
%        \item S2-5: similar, max at 6-8 km
%        \item \todo{say more!}
%    \end{itemize}
%\end{itemize}

\todo{switch order of subfigs? Go number of clouds, MF/cld, Total MF as this is the order it's most logical to talk about them in?}

Figure \ref{fig:mf_profile} shows, from left to right, profiles of the mass flux per cloud, the number of clouds and the total convective mass flux. Clouds are diagnosed as set out in Section \ref{subsec:cloud_diag}. The leftmost profile shows that there is a distinct difference between the two unorganized experiments, S0 and S1, and the organized experiments, S2-5. The organized experiments all show an increase in mass flux per cloud as height increases, reaching maximum values at around \SI{10}{km}. This can be explained by noting that the number of clouds is approximately constant for these experiments, and therefore the latent heat released in the updraught acts to strengthen the total mass flux in those plumes. In contrast, the mass flux per cloud in S0 and S1 is approximately constant over the troposphere. There are two balancing effects in this case - there are fewer clouds at higher heights, and each cloud that makes it to a certain height will be stronger due to the reason above. Similar behaviour was seen in \cite{PC2008}. Mass flux per cloud increases as the shear increases. This could indicate that the shear has an inhibiting effect on convection, and so a larger and stronger cloud must form in order to overcome this inhibition.

\begin{figure}[hbp!]
    \centering
    \includegraphics[width=400px]{figs/{atmos.profile_plot.mf_profile}.png}
    \caption{}
    \label{fig:mf_profile}
\end{figure}

From the number of clouds in the middle figure in Figure \ref{fig:mf_profile}, there is again a clear relationship - above \SI{1}{km}, as the shear increases, the number of clouds decreases. The clouds also start forming at a lower height for the strongly sheared experiments. There is, once again, a distinction between the behaviour of the unorganized and organized cases. In the unorganized experiments, the number of clouds decreases with increasing height, with a clearly defined maximum at \SI{1.5}{km}. What is conspicuous by its absence is a secondary maximum in the number of clouds profile for the organized experiments. When compared to the cloud liquid water in Figure \ref{fig:thermodyn_profile}, this shows that the cloud diagnosis is effectively filtering out this region of stratiform cloud. This supports the assertion that the maximum in cloud liquid water is caused by a region of trailing stratiform cloud.

In the total convective mass flux, equal to the product of the mass flux per cloud and the number of clouds, S0 is seen to have the largest absolute value. This implies that the gradient $\frac{d \theta}{dz}$ must be least for this experiment \todo{lay the groundwork for this assertion earlier - heating is due to subs. and $mf \times \frac{d \theta}{dz}$ is proportional to the heating}. The other experiments all show more closely grouped values, with the organized experiments increasing in the upper troposphere, with a maximum value between \SI{6}{}-\SI{8}{km}. \todo{what more can I say about this?}

\subsection{Mass flux variability}

%\subsection{Spatial variability}
%\begin{itemize}
%    \item Figure 2: Effects of spatial lengths on total mass-flux (c.f. \cite{PC2008}, Fig. 1).
%    \item Figure 3: Effects of spatial lengths on PDF of mass-flux per cloud cell (c.f. \cite{CC2006II}, Fig. 2).
%    \item N.B. perhaps combined into one figure.
%\end{itemize}

%\begin{itemize}
%    \item \textbf{Figure \ref{fig:mf_hists}:}
%    \item \textbf{upper:}
%    \begin{itemize}
%        \item this shows hist of $m$: $p(m) \times N_{cld}$
%        \item exponential decrease for S0, cf. \ref{eqn:pdf_mass_flux_per_cloud}
%        \item approx. exponential decrease for S1-5, but: fast drop at start? 2 slopes with break at \SI{1.25e8}{kg.s^{-1}}
%        \item S2-5: all similar
%    \end{itemize}
%    \item \textbf{lower:}
%    \begin{itemize}
%        \item this shows how much a given mf contributes to the total mf: $m \times p(m) \times N_{cld}$
%        \item as S up, max. contrib. comes from larger mf clds
%        \item S2-5 similar (again)
%    \end{itemize}
%\end{itemize}
%
\subsubsection{Mass flux per cloud}
To test Equation \ref{eqn:pdf_mass_flux_per_cloud}, we present a histogram of the mass flux per cloud (Figure \ref{fig:mf_hists}, upper figure), at a height of \SI{2}{km}. We find excellent agreement with this PDF for the S0 case, with an exponential decrease in the number of clouds with a given mass flux. \todo{quantify this - fit a line}. This results matches previous findings by \cite{CC2006II}, cf. their Figure 2. The cloud diagnosis we are using is slightly different from the one they used: they used either a $w$ threshold or a cloud liquid water threshold; we use both together. They also performed their analysis at a different height: \SI{2.4}{km} instead of \SI{2}{km}. We find that performing the same analysis at \SI{1.5}{km} and \SI{2.5}{km} produces qualitatively similar results, although the exact number of clouds is different (not shown).

\begin{figure}[hbp!]
    \centering
    \includegraphics[width=400px]{figs/{atmos.mass_flux_plot.z1_both}.png}
    \caption{}
    \label{fig:mf_hists}
\end{figure}


For the sheared experiments, there is still good agreement with this exponential PDF. \todo{quantify} . The organized experiments all show similar behaviour, with the gradient of each of these experiments all having a similar value. The number of clouds with low mass fluxes decreases with increasing shear, consistent with Figure \ref{fig:mf_profile}, centre. The maximum mass flux increases with increasing shear, consistent with Figure \ref{fig:mf_profile}, left.

Figure \ref{fig:mf_hists} also shows a measure of how much clouds with a given mass flux contribute to the total mass flux ($m \times p(m) \times N_{cld}$, where $m$ is the mass flux per cloud, $p(m)$ is its PDF and $N_{cld}$ is the total number of clouds). From this it is possible to see which mass flux per cloud has the largest effect on the total mass flux. For S0, the clouds that have the largest effect are the ones with the lowest mass flux, i.e. there are a large number of clouds with a small mass flux. The maximum contribution increases with increasing shear; higher shear leads to larger clouds in terms of mass flux per cloud. \todo{this para is a little weak and not sure it adds much?}

%\begin{itemize}
%    \item \textbf{Figure \ref{fig:mf_spatial_hists}:}
%    \item \textbf{upper: S0: n=1, 2, 4}
%    \begin{itemize}
%        \item shows increase in variance with S incr
%        \item \todo{How to explain n=4 kink at low mf?}
%    \end{itemize}
%    \item \textbf{lower: n=1: S0-5}
%    \begin{itemize}
%        \item decrease in mean mf with s incr.
%        \item S2-5 similar
%        \item note, there are far fewer clds in the domain for higher S, but width of S0 vs e.g. S4 is similar
%        \item this means that variance of S2-5 is (relatively) lower (I think)
%    \end{itemize}
%\end{itemize}
%
\subsubsection{Mass flux at different spatial scales}
We also want to know the variance of the different experiments at different spatial scales. To do this, we subdivide the domain into an equal number of square subregions, with lengths 1, 2 and 4 times smaller than the length of the domain. Each of these has respectively subregions with an area of \SI{256}{} $\times$ \SI{256}{km^2}, \SI{128}{} $\times$ \SI{128}{km^2}, \SI{64}{} $\times$ \SI{64}{km^2}. We then compute the histogram of the total convective mass flux within each subregion, the results can be seen in Figure \ref{fig:mf_spatial_hists}. The analysis is again carried out at \SI{2}{km}, although similar quantitative results are seen at \SI{1.5}{km} and \SI{2.5}{km} (not shown). The top figure shows the histogram for the S0 experiment over three different spatial scales. As was found in a previous study, the variance increases when the area of the subregion decreases \parencite{PC2008}. It should be noted that in their Figure 1, the prescribed cooling that they were using was far higher, \SI{16}{K.day^{-1}}, as opposed to \SI{2}{K.day^{-1}} used here. They therefore see far more clouds in their domain, which leads to a much narrower distribution. The qualitative results remain the same though, as they decrease the size of their subregions, they see an increase in variance.

\begin{figure}[hbp!]
    \centering
    \includegraphics[width=400px]{figs/{atmos.mass_flux_spatial_scales_plot.both_z1}.png}
    \caption{}
    \label{fig:mf_spatial_hists}
\end{figure}


In the lower figure in Figure \ref{fig:mf_spatial_hists}, we compare the total convective mass flux across the whole domain for our different experiments. Note that the blue dashed line is the same in both upper and lower figures. As the shear is increased, the total mass flux in the domain decreases, with the organized experiments all showing similar distributions. The width of the distributions across all the experiments is broadly the same. \todo{Don't fully understand this, can we discuss? Thought the variance increase would be bigger.} %This coupled with the fact that there are far fewer clouds in the high shear experiments, means that the relative variance \todo{not sure how to phrase this} is actually lower.

\subsection{Cloud field organization}

\begin{itemize}
    \item Figure \ref{fig:org}: Measure of organization demonstrating effectively the higher organization in the organized convection cases.
\end{itemize}

\begin{figure}[hbp!]
    \centering
    \includegraphics[width=400px]{figs/{atmos.org_plot.z1_combined_log}.png}
    \caption{}
    \label{fig:org}
\end{figure}

\subsection{Momentum transports}
\begin{itemize}
    \item The momentum flux can be calculated using $\rho \mean{u' v'} = \rho \frac{u_{inc}}{\Delta t} \Delta z$.
    \item \todo{keep?} Figure \ref{fig:momf_profile}: Momentum flux profiles for different experiments (c.f. \cite{RE2001}, Fig. 7).
\end{itemize}

\begin{figure}[hbp!]
    \centering
    \includegraphics{figs/{atmos.profile_plot.momentum_flux_profile}.png}
    \caption{}
    \label{fig:momf_profile}
\end{figure}


\section{Conclusions}

%\newpage
%\subsection*{Main papers (*:key papers)}
%\begin{itemize}
%    \item * \cite{CC2006I}
%    \item * \cite{CC2006II}
%    \item * \cite{RE2001}
%    \item * \cite{RKW1988}
%    \item * \cite{PC2008}
%    \item \cite{birch2014scale}
%    \item \cite{cohen2004response}
%    \item \cite{gregory1997parametrization}
%    \item \cite{houze1977structure}
%    \item \cite{kershaw1997parametrization}
%    %\item \cite{parker2007simulated}
%    \item \cite{robe1996moist}
%    \item \cite{sakradzija2016stochastic}
%    \item \cite{sengupta1990cumulus}
%    \item \cite{TMM1982}
%\end{itemize}
%
%
\printbibliography[title={References}]


\end{document}
